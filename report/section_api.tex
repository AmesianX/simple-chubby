\section{User API}
\label{section:api}

SimpleChubby user API consists of
a file system interface for serialized file access and modification,
a set of lock primitives,
and an event subscription interface.
%SimpleChubby provides an interface similar to the Chubby API~\cite{burrows2006chubby},
%which exports a file system interface similar to, but simpler than that of UNIX.
The underlying file system is organized similar to Unix,
which consists of a strict tree of files and directories.
Both files and directories are called \emph{nodes}.
%All the operations from all the clients are serialized at the leader server.

\begin{figure}
\centering
\begin{lstlisting}[language=C++,
    basicstyle=\footnotesize\ttfamily,
    commentstyle=\bfseries,
    deletekeywords={delete},
    morekeywords={FileHandlerId, string, Mode, FileContent, MetaData, ChubbyEvent, EventCallback}]
/* File and directory operations */
FileHandlerId fileOpen(const string &fname,
        Mode mode);
void fileClose(FileHandlerId fdId);
bool fileDelete(FileHandlerId fdId);

/* Content read and write */
bool getContentsAndStat(FileHandlerId fdId,
        FileContent *content,
        MetaData *data);
bool setContents(FileHandlerId fdId,
        const FileContent &content);

/* Lock operations */
void acquire(FileHandlerId fdId);
bool tryAcquire(FileHandlerId fdId);
void release(FileHandlerId fdId);

/* Event subscription */
void registerEvent(ChubbyEvent e,
        EventCallback cb);
void deleteEvent(ChubbyEvent e);
\end{lstlisting}
\caption{SimpleChubby API}
\label{fig:api}
\end{figure}

The full list of the API is shown in Figure~\ref{fig:api}.
An application calls \texttt{fileOpen()} to retrieve a file handler
associated with an opened node. Similar to UNIX's implementation,
this function returns an integer type of identifier for each file handler,
which is used by all other functions in the API.
We simplify the \texttt{fileOpen()} interface by removing the argument of
the parent handler, and always specify an absolute path of a node.
The \texttt{mode} argument in the \texttt{fileOpen()} consists of a set of flags,
including \texttt{READ}/\texttt{WRITE} mode, and create \texttt{NEW\_DIRECTORY} flag.
By setting \texttt{EPHEMERAL} flag, a client can create ephemeral nodes,
which will be automatically removed when the connection fails.

Clients may also subscribe to a range of events during \texttt{fileOpen()},
including ``file contents modified'' and ``lock state changed'',
using the \texttt{mode} arguement.
The events are delivered to the clients asynchronously.
The event handler callback functions are supplied and registered by user
applications beforehand, and will be called by the client library when
corresponding events are received.

Our API supports a simplified failover model of Chubby. The server failures
are transparent to user applications if the connection can be reconstructed
within a certain grace period, otherwise an exception will be exposed to
the applications and all the locks it holds are assumed to be invalid.
If a client fails to connect to the leader server, its holding locks
will be reclaimed after a certain timeout.

The user applications can use this API to perform various synchronization
jobs. We provide three examples in Section~\ref{section:apps}.


