%\subsection{Terminology}
%
%\begin{itemize}[noitemsep]
%  \item Users.
%  Programmers who use SimpleChubby user-level interfaces to write a user-defined synchronization service.
%  \item Servers.
%  SimpleChubby service runs on a set of processes called servers.
%  \item Clients.
%  A user-written program runs on a process called a client, which connects to the server
%  and cooperates with the server to implement SimpleChubby lock service.
%  \item Leader server.
%  One of servers is responsible for processing requests from the clients, and coordinates other servers.
%  \item Follower server.
%  The servers other than the leader servers.
%\end{itemize}

SimpleChubby implementation consists of a client library linked to user programs,
and a server executable running on each server.
When user code invokes the client library interfaces to access files or perform lock operations,
the client library initializes corresponding remote procedure calls (RPCs) to the leader server,
waits for replies, and returns the result back to the user.

\paragraph{Client side.}
The client library offers a set of interfaces for users to access SimpleChubby
lock service. It sends user requests to the leader server through PRCs,
and receives any events from the leader server.
The client library also maintains a set of states, including opened file handlers,
subscribed events, and outstanding lock acquire requests.
In addition to sanity checks, it uses this information to help server failover.
Once an older leader fails, the client library tries to connect to a new leader,
and send its states which are necessary to reconstruct server's non-persistent
states.

%User code does not talk with the leader server directly since the client library
%needs to know what requests are issued so that it can help a leader server
%to recover lost states during a server fail-over.
%Meanwhile, the client library can do certain sanity checks before sending requests
%to the leader server.

\paragraph{Server side.}
SimpleChubby runs on a group of servers to provide a reliable lock service.
A server runs as either the leader server or a follower server,
decided by a consensus protocol.
When running as a follower server, it serves as a replica in the consensus
protocol, and prepares to be promoted to a leader server.

When running as a leader server, it processes requests from clients and send
events to clients.
The leader server also monitors client failures, and clears the states associated
with the failed clients.
The leader server maintains two categories of states: persistent states and
in-memory states.
The persistent states, including lock ownership and file system content,
are replicated among replicas, and thus accesses or updates to them go
through the consensus protocol.
The rest of states are not replicated through the consensus protocol,
and thus will be lost after a leader change.

SimpleChubby does not make all the leader server states persistent because
of the overhead of the consensus protocol.
Instead, non-persistent states can be recovered using client side information.
We talk about the failover mechanism in Section~\ref{section:fail}.
