\section{Introduction}

\subsection{Terminology}

\begin{itemize}
  \item Users.
  People who uses SimpleChuby user-level interfaces to write a synchronization service.
  \item Servers.
  SimpleChubby service runs on a set of processes called servers.
  \item Clients.
  A user-writen program runs on a process called a client, which connects to the server
  and cooperates with the server to implement a synchronization service.
  \item Leader server.
  \item Follower server.
\end{itemize}

\subsection{System architecture}

SimpleChubby implementation consists of a client library linked to user programs,
and a server executable running on each server.

The client library offers a set of interfaces for users, and talks with the leader
server to implement the interfaces. 

The user does not talk with the leader server directly since the client library
needs to know what requests the user issues so that it can help a leader server
to recover lost states when server failover.
Meanwhile, the client library can do certain sanity checks before sending requests
to the leader server.

When a user invokes the interfaces to access files or perform locking operations,
the client library performs corresponding remote procedure calls to the leader server,
waits for replies from the leader server, and returns the result back to the user.
A user can also register certain events,
such as a file modification event or a lock ownership change event, 
in the client library,
and it will be notified when the registered events happen.
Event notification happens asynchonized with other user interface calls.

The client library also cooperates with the servers for server failover.  
It will try to reconnect to a new leader server once the old one fails,
and help the new server recover lost states,
such as opened files, and lock waiting queues.

The server executable can run as the leader server or as a follower server,
as decided by the Viewstamp consensus protocol.

When running as a follower server, it serves as a replica to duplicate critical
operations happening on the leader server, and prepares to be promoted to a leader
server.

When running as a leader server, it accepts and processes requests from clients.
The leader server also monitors client failures, and clears its states after failure,
i.e., reclaiming its holding locks after a certain timeout.

The leader server maintains two catogeries of states.
The first catogery of states includes lock ownership and file system content. These
states are made persistent, and accesses or changes to them go through the consensus
protocol.
The second catogery of states includes open client lists, lock waiting queues,
and open file handlers.
Theses states can be recovered with the help of clients,
and thus are not persistent without going through consensus protocol.

SimpleChubby does not make all the leader server states persistent by the consensus
protocol, because some states change much often but easy to be recovered,
and list or queue data structures are not easy to be stored into SQL database.
 
\subsection{Details about the client}
To be simple, and avoid overwhelming the server, the client process will block until a remote procedure call is
replied unless the leader server fails. That is, only one outstanding remote procedure call is allowed between
a client and the leader server.

Cite: \cite{nothing}
