\section{Fault Tolerance}
\label{section:failure}

\subsection{Client Failure}

During a client failure, the leader server releases all the locks
previously held by the failed client after a certain timeout.
This guarantees that locks are eventually released and no deadlock happens.

For garbage collection
purpose, the server also deletes all the file handlers opened by the client,
outstanding lock acquire requests sent by the client, and all the registered
events from the client.

Each client maintains a TCP connection with the leader server,
which is called a session.
The session is dead if either end shuts down the TCP connection.
And the leader server will assume a client failure once the session is dead.
Notice that this is a simplified implementation.
If a client dies before explicitly shutting down the TCP connection,
the client failure cannot be detected.

\subsection{Server Failover}

When the leader server fails, the Viewchange protocol elects a new leader. After
the leader change, all persistent data remain consistent. However, the in-memory
data structures at the new leader need to be reconstructed. 

Two methods are used for the reconstruction.
First, a client eagerly sends some of its states to the new leader after the
connection is established, which helps recover the in-memory structures of
the leader, including the queues of outstanding lock acquire requests and
the lists of event subscriptions from clients. In particular, the client library
keeps a copy of subscribed events and its outstanding lock acquire request.
After a leader change, a client connects to the new leader, sends
reopen requests containing the previously subscription of events, and finally
send an \texttt{acquire()} request if there is any outstanding request.
On the other side, the leader accepts any reopen requests with valid file
handlers, and add the event subscriptions to its in-memory data structure.

However, the in-memory data structure of file handlers does not need to be
reconstructed eagerly. The second method we use is to lazily recover the
the file handler at the server side during the normal case of processing
SimpleChubby requests.
If the new leader receives a request with a file handler that
missing in the leader's memory, it checks the signiture of the file
handler and initiates a reopen. If the reopen succeeds (i.e. the node exits
and the meta data matches), the leader add this file handler into its memory.
