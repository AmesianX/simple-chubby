\section{User API}
\label{section:api}

The client provides an interfaces identical to the Chubby API~\cite{burrows2006chubby}.
We support file or directory operations, content
read and write, lock operations, and event handler registration.
The full list of the API is shown in Figure~\ref{fig:api}.

\begin{figure}
\centering
\begin{lstlisting}[language=C++,
    basicstyle=\footnotesize\ttfamily,
    commentstyle=\bfseries,
    deletekeywords={delete},
    morekeywords={FileHandlerId, string, Mode, FileContent, MetaData, ChubbyEvent, EventCallback}]
/* File and directory operations */
FileHandlerId open(const string &fname,
        Mode mode);
void close(FileHandlerId fdId);
bool delete(FileHandlerId fdId);

/* Content read and write */
bool getContentsAndStat(FileHandlerId fdId,
        FileContent *content,
        MetaData *data);
bool setContents(FileHandlerId fdId,
        const FileContent &content);

/* Lock operations */
void acquire(FileHandlerId fdId);
bool tryAcquire(FileHandlerId fdId);
void release(FileHandlerId fdId);

/* Event handler registration */
void registerCallback(ChubbyEvent e,
        EventCallback cb);
void deleteCallback(ChubbyEvent e);
\end{lstlisting}
\caption{SimpleChubby API}
\label{fig:api}
\end{figure}

Applications use \texttt{Open()} to create a handler associated with a file
or directory handler, and the other functions all operate on the handler.
The \texttt{mode} argument in the \texttt{Open()} function specifies the read
or write mode of the handler, as well as the events to which this client subscribes.
The event handler callback functions are supplied by user applications and
registered beforehand, which will be called by the client library when
corresponding events are received.

The user applications can use this API to perform various synchronization
jobs. We provide two examples in Section~\ref{section:apps}.


