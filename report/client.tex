The client library of SimpleChubby is responsible for sending RPCs to the
leader server, and returning the results to the user-level applications.
To avoid overwhelming the server, the client process will block until a RPC
gets replied. This implies that only one
outstanding RPC is allowed between a client and the leader server.

In addition, the client library needs to
monitor event messages from the server and execute up-call
to the user level.
Since a user program might be blocked by such operations
as \texttt{acquire()}, events have to be delivered asynchronized
to guarantee that events are delivered in time.

The event subscription is necessary
for the applications we implemented in Section~\ref{section:apps},
because lock primatives can not notify a client about new events,
such as leader change or content updated.

Finally, when the leader server fails, the client library needs to figure out
the new leader after server-side view change, and helps to recover some
non-persistent states at the server side. We will talk about the details in
Section~\ref{section:failure}.



%The leader only replicates and persistently keeps part of its state for
%simplicity, as described in Section~\ref{TODO}. When failure happens, some
%states, including the lock waiting queues and the event subscribing lists,
%will be lost and can't be automatically recovered at the new leader. Therefore,
%the client library needs to keep this information locally. When it connects
%to the new leader server, it needs to re-send the \texttt{acquire()} RPC,
%and tells the new leader all the events it subscribes.

